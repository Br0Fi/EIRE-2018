% Autor: Leonhard Segger, Alexander Neuwirth
% Datum: 2017-10-30
\documentclass[
	% Papierformat
	a4paper,
	% Schriftgröße (beliebige Größen mit „fontsize=Xpt“)
	12pt,
	% Schreibt die Papiergröße korrekt ins Ausgabedokument
	pagesize,
	% Sprache für z.B. Babel
	ngerman
]{scrartcl}

% Achtung: Die Reihenfolge der Pakete kann (leider) wichtig sein!
% Insbesondere sollten (so wie hier) babel, fontenc und inputenc (in dieser
% Reihenfolge) als Erstes und hyperref und cleveref (Reihenfolge auch hier
% beachten) als Letztes geladen werden!

\usepackage{tikz}
\usetikzlibrary{calc,patterns,angles,quotes} % loads some tikz extensions\usepackage{tikz}
\usetikzlibrary{babel}

% Silbentrennung etc.; Sprache wird durch Option bei \documentclass festgelegt
\usepackage{babel}
% Verwendung der Zeichentabelle T1 (Sonderzeichen etc.)
\usepackage[T1]{fontenc}
% Legt die Zeichenkodierung der Eingabedatei fest, z.B. UTF-8
\usepackage[utf8]{inputenc}
% Schriftart
\usepackage{lmodern}
% Zusätzliche Sonderzeichen
\usepackage{textcomp}

% Mathepaket (intlimits: Grenzen über/unter Integralzeichen)
\usepackage[intlimits]{amsmath}
% Ermöglicht die Nutzung von \SI{Zahl}{Einheit} u.a.
\usepackage{siunitx}
% Zum flexiblen Einbinden von Grafiken (\includegraphics)
\usepackage{graphicx}
% Abbildungen im Fließtext
\usepackage{wrapfig}
% Abbildungen nebeneinander (subfigure, subtable)
\usepackage{subcaption}
% Funktionen für Anführungszeichen
\usepackage{csquotes}
\MakeOuterQuote{"}
% Zitieren, Bibliografie
\usepackage[sorting=none]{biblatex}


% Zur Darstellung von Webadressen
\usepackage{url}
%chemische Formeln
\usepackage[version=4]{mhchem}
% siunitx: Deutsche Ausgabe, Messfehler getrennt mit ± ausgeben
\usepackage{floatrow}
\floatsetup[table]{capposition=top}
\usepackage{float}
% Verlinkt Textstellen im PDF-Dokument
\usepackage[unicode]{hyperref}
% "Schlaue" Referenzen (nach hyperref laden!)
\usepackage{cleveref}
\sisetup{
	locale=DE,
	separate-uncertainty
}
%\bibliography{6Mi_M3_29-11-2017_References}
%TODO anpassen

\begin{document}
	
	\begin{titlepage}
		\centering
		{\scshape\LARGE Protokoll zu \par}
		\vspace{1cm}
		{\scshape\huge Einführung in rechnergestütztes Experimentieren \par}
		\vspace{3cm}
		
		{\large Jannik Tim Zarnitz (E-Mail: j\_zarn02@wwu.de) \par}
		{\large Leonhard Segger (E-Mail: l\_segg03@wwu.de) \par}
		\vfill
		
		in der Woche 03.09.2018 bis 06.09.2018\par
		betreut von\par
		{\large Dr. Jürgen Berkemeier}
		
		\vfill
		
		{\large \today\par}
	\end{titlepage}
	\tableofcontents
	\newpage

	\section{Tag 1} \label{Tag 1}
	
	\subsection{Aufbau einer Sinus- bzw. Bessel-Funktion}
	
	
	
	\subsection{Lissajous-Figuren}
	
	
	
	\section{Tag 2} \label{Tag 2}
	
	\subsection{Realisierung eines digitalen Oszilloskops}
	Es wird ein Funktionsgenerator verwendet.
	Dessen Signal wird über einen Analog-Digital-Wandler durch den Computer erfasst.
	
	
	\subsection{Fouriertheorem}
	
	\subsection{Abtasttheorem}
	%TODO kurze Zusammenfassung seiner Theorie-Sachen, hab ich auf Papier
	
	\section{Tag 3} \label{Tag 3}
	%eig. zunächst Fertigstellung von Tag2 Teil 1
	
	\subsection{Leakage-Effekt und Fensterfunktion}
	
	\subsection{amplitudenmodulierte Signale} %groß?
	
	\subsubsection{Erzeugung}
	% + Ausgabe über Soundkarte
	
	\subsubsection{Demodulation}
	%Betrag, Quadrat
	
	\section{Tag 4} \label{Tag 4}
	
	\subsection{Demodulation eines AM-Signals mittels Trägerfrequenzmultiplikation}
	
	\subsection{Erzeugung eines phasen- bzw. frequenzmodulierten Signals}
	
	\subsection{Demodulation eines phasen- bzw. frequenzmodulierten Signals}
	
	\subsubsection{Erweiterte Demodulation mit Bandpass und zusätzlicher Integration des Signals}
	
	
	%\printbibliography
\end{document}
